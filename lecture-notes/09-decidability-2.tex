%template1.tex
%The following LaTeX source file represents the simplest kind of slide presentation; no overlays, no included graphics. Substitute your favorite style for ``pascal''. To create the PDF file template1.pdf, (1) be sure to use the prosper class, then (2) execute the command latex template1.tex, and (3) the command dvipdf template1.dvi.

%%%%%%%%%%%%%%%%%%%%%%%%%%%%%%% template1.tex %%%%%%%%%%%%%%%%%%%%%%%%%%%%%%%%%%%
\documentclass[a4paper,blends,pdf,colorBG,slideColor]{prosper}
% definitions for slides for CSC544
% Lutz Hamel, (c) 2007

\hypersetup{pdfpagemode=FullScreen}

\usepackage{amssymb}
\usepackage{latexsym}
\usepackage{amsmath}
%\usepackage[usenames]{color}
\usepackage{xypic}


\newcommand{\term}[1]{\ensuremath{\mbox{\bf #1}}}
\newcommand{\nonterm}[1]{\ensuremath{\mbox{#1}}}
\newcommand{\ifstmt}[3]{\ensuremath{{\bf if}\; {#1}\;{\bf then}\;{#2}\;{\bf else}\;{#3}\;\term{end}}}
\newcommand{\whilestmt}[2]{\ensuremath{{\bf while}\; {#1}\;{\bf do}\;{#2}\; \term{end}}}
\newcommand{\funcstmt}[3]{\ensuremath{{\bf fun}\; {#1}\; {\bf is}\; {#2} \; {\bf return}\; {#3}}}
\newcommand{\syntaxset}[1]{\ensuremath{\mbox{\bf #1}}}
\newcommand{\orbar}{\;|\;}
\newcommand{\bs}[1]{\begin{slide}{#1}\ptsize{8}}
\newcommand{\es}{\end{slide}}
\newcommand{\co}{\,\colon\;}
\newcommand{\pair}[2]{\ensuremath{\langle {#1}, {#2} \rangle}}
\newcommand{\encode}[1]{\ensuremath{\langle {#1} \rangle}}
\newcommand{\mytab}{\makebox[.15in]{}}
%\newcommand{\abs}[1]{{\mid{#1}\mid}}
\newcommand{\abs}[1]{{|{#1}|}}
\newcommand{\ol}[1]{\overline{#1}}

\newcommand{\qaccept}{\ensuremath{q_{\mbox{\tiny accept}}}}
\newcommand{\qreject}{\ensuremath{q_{\mbox{\tiny reject}}}}
\newcommand{\accept}{{\em accept}}
\newcommand{\reject}{{\em reject}}

\newcommand{\machine}[1]{
	\begin{quote}
	{#1}
	\end{quote}
	}

\newcommand{\fdef}[1]{
	\begin{center}
	\fbox{
	\begin{minipage}{3.5in}
	{\bf Definition:}
	{#1}
	\end{minipage}
	}
	\end{center}
	}

\newcommand{\ftheorem}[1]{
	\begin{center}
	\fbox{
	\begin{minipage}{3.5in}
	{\bf Theorem:}
	{#1}
	\end{minipage}
	}
	\end{center}
	}

\newcommand{\flemma}[1]{
	\begin{center}
	\fbox{
	\begin{minipage}{3.5in}
	{\bf Lemma:}
	{#1}
	\end{minipage}
	}
	\end{center}
	}


\newcommand{\fframe}[1]{
	\begin{center}
	\fbox{
	\begin{minipage}{3.5in}
	{#1}
	\end{minipage}
	}
	\end{center}
	}

\newcommand{\nframe}[1]{
	\begin{center}
	\begin{minipage}{3.5in}
	{#1}
	\end{minipage}
	\end{center}
	}

\begin{document}

\bs{Another Proof}

We just saw a  proof that languages that are not Turing-recognizable exist based on
the fact that a decider for $A_{TM}$ cannot exist

Let's look at another proof that shows that some languages are not algorithmic.

The proof proceeds by showing that the set of all Turing machines is countably infinite whereas the set of all languages is uncountable.  Therefore, there exist some languages
that are not recognized by a Turing machine.

{\bf NOTE:} Let $\aleph_0$ be the cardinality of the natural numbers and $C$ the cardinality of the reals, then Cantor's continuum hypothesis states that $\aleph_0 < C$.
That is, the natural numbers are {\em countably infinite} whereas the reals are {\em uncountable}.\footnote{The book has the classical proof of the uncountability of the reals based on diagonalization.}

\begin{center}
\em ``There are fewer natural numbers than there are reals.''
\end{center}
\vspace{.5in}
\es

\bs{Countable Sets}

Here are some simple examples of countable sets:



{\scriptsize
\begin{minipage}[t]{1in}
\[
\begin{array}{l | l}
n & f(n) \\ \hline
1 & 1\\
2 & 2 \\
\vdots & \vdots\\
k & k
\end{array}
\]
\end{minipage}
\begin{minipage}[t]{1in}
\[
\begin{array}{l | l}
n & f(n) \\ \hline
1 & 10\\
2 & 20 \\
\vdots & \vdots\\
k & k*10
\end{array}
\]
\end{minipage}
\begin{minipage}[t]{1in}
\[
\begin{array}{l | l}
n & f(n) \\ \hline
1 & 2\\
2 & 4 \\
\vdots & \vdots\\
k & k*2
\end{array}
\]
\end{minipage}
\begin{minipage}[t]{1in}
\[
\begin{array}{l | l}
n & f(n) \\ \hline
1 & 1\\
2 & 3 \\
\vdots & \vdots\\
k & k*2 - 1
\end{array}
\]
\end{minipage}
}



{\bf Observation:} In all cases the mapping $f$ between $n$ and $f(n)$ is one-to-one
and onto, that is, it is bijective: Each value of $n$ uniquely identifies a value of $f(n)$ and there is no way to construct a member of the codomain
of $f$ that does not already appear in the correspondence.

{\bf Observation:}  If we choose $k=\infty$ then we call the set $f(n)$ countably infinite.
\es

\bs{The Reals are Uncountable}
{\small
\fframe{\begin{center}{\bf Proposition:} The set of all reals is uncountable.\end{center}}
}
{\scriptsize
{\bf Proof:} By contradiction.  Assume that the
set of all reals is countable.  Let $f: {\mathbb N} \rightarrow {\mathbb R}$ be a bijective mapping
from the naturals to the set of all reals.  Only consider the reals in the interval $[0,1]$. Then by assumption we have,
\begin{center}
{\tiny
\begin{minipage}[t]{1in}
$
\begin{array}{l | l}
n & f(n) \\ \hline
1 & 0.{\color{red}3}156978\ldots\\
2 & 0.6{\color{red}5}39879\ldots \\
3 & 0.11{\color{red}3}4768\ldots\\
\vdots & \vdots\\
k & 0.2200354653\ldots{\color{red}5}\ldots\\
\vdots & \vdots\\
\end{array}
$
\end{minipage}
}
\end{center}

That is, since $f$ is a bijective mapping we have a list of all possible real values.
We now show that we can construct a real value that is not included
in the list above.  We construct this value
by taking the $i$th digit after the decimal point for each
real value identified by $i$ appearing in the correspondence and adding one to it (modulo 10).  We use these
newly generated digits to construct a new value in $[0,1]$: $0.464\ldots6\ldots$

By construction this value differs from any real value appearing in the codomain of
the mapping by at least  one digit.  This means that there is at least
one value in the codomain of $f$ which is not in the image of $f$.  This is a contradiction,
since $f$ was assumed to be a bijective mapping.  Therefore, our assumption
that the set of all reals is countable must be wrong. $\Box$
}
\es

\bs{Diagonalization}
The general proof technique is as follows:
\begin{enumerate}
\item Assume that you have a correspondence $f\co {\mathbb N} \rightarrow {\cal S}$,
where ${\cal S}$ is the structure you want to investigate.
\item Construct a grid with the rows containing elements of your correspondence.
\item Now construct a new item of your structure as an element of $\cal S$
such that it will differ from all other elements in the correspondence on the
major diagonal.
\item This is a contradiction since you have constructed an element in $\cal S$ not listed in the
correspondence, therefore, $f$ is not a correspondence and the structure $\cal S$
is not countable.
\end{enumerate}
\es

\bs{Turing Machines are Countably Infinite}
\fframe{
{\bf Theorem:} The set of all Turing machines is countably infinite.
}

{\bf Proof:}  Let $\Sigma_{0,1} = \{ 0,1\}$ be the alphabet over the symbols $0$ and $1$, observe that the
set of all strings over this alphabet, say $\Sigma_{0,1}^*$, is countably infinite by the fact that
we can interpret each string in this set as the binary encoding of a natural number.
Now, let $\encode{M}_{0,1}$ be the binary encoding of some Turing machine $M$.  
It is clear that such an encoding exists, since any other encoding $\encode{M}_{\Sigma'}$
can be transformed into the encoding $\encode{M}_{0,1}$ by representing each
symbol in $\Sigma'$ as a unique string in $\Sigma_{0,1}$.\footnote{We do this everyday on our digital computers.}
Now, let $\encode{M}_{0,1}^*$ be the set of all encoded, valid Turing machine descriptions.
Observe that  $\encode{M}_{0,1}^*\subseteq \Sigma_{0,1}^*$.
This implies that the set of all Turing machines is countable infinite.$\Box$

\vspace{.5in}
\es


\bs{\large Infinite Binary Sequences are Uncountable}
{\small
\fframe{{\bf Theorem:} The set of all infinite binary sequences is uncountable.}
}
{\scriptsize
{\bf Proof:} We prove this by contradiction using the diagonalization method.  Assume that we can construct
a bijective mapping $f\co {\mathbb N} \rightarrow {\cal B}$, where $\mathbb N$ are the natural
numbers and $\cal B$ is the set of all infinite binary sequences.  Then,

{\tiny
\begin{center}
\begin{tabular}{r | l}
$n$ & $f(n)$ \\ \hline
$1$ & ${\color{red}0}100111\ldots$\\
$2$ & $1{\color{red}1}111000\ldots$\\
$3$ & $10{\color{red}1}1001\ldots$\\
$\vdots$ & $\vdots$\\
$k$ & $0010\ldots{\color{red}0}_k\ldots$\\
$\vdots$ & $\vdots$\\
\end{tabular}
\end{center}
}
Observe that we can always construct another binary sequence which will differ 
from all the enumerated sequences by at least one bit,
\[
{\color{red} 100}\ldots{\color{red}1_k}\ldots
\]
That is, for any value $i$ we have constructed a sequence which will differ in 
value from $f(i)$ in the $i^{\mbox{th}}$ bit.  Therefore, there exist
elements in $\cal B$ that are not in the image of $f$.  That means
our assumption that $f$ is bijective is incorrect. $\Box$
}
\es

\bs{Languages are Uncountable}
{\small
\fframe{{\bf Theorem:} The set of all languages $L$ over alphabet $\Sigma_{0,1}$
is uncountable.}

{\bf Proof:} We show this by constructing a bijective mapping $f\co L \rightarrow {\cal B}$.
For each language $A \in L$ we can construct a unique element in ${\cal B}$ called
the {\em characteristic sequence}.  Let $\Sigma_{0,1}^* = \{ s_1, s_2,s_3,\ldots\}$, then
the $i$th bit of the characteristic sequence of $A$ is $1$ if $s_i \in A$ and $0$ if
$s_i \not\in A$. Note,
\begin{itemize}
\item The empty language has the characteristic sequence $000000\ldots$
\item The language $\Sigma_{0,1}^*$ has the characteristic sequence $1111\ldots$
\end{itemize}
The mapping $f$ is bijective in that any possible language in $L$ has a unique sequence
in $\cal B$ and any sequence in $\cal B$ uniquely defines a language in $L$.
$\Box$
}
\es

\bs{\large Not Turing-recognizable Languages}

\fframe{{\bf Theorem:} Some languages are not Turing-recognizable.}

{\bf Proof:} Observe that $\#\encode{M}_{0,1}^* \le \aleph_0$ and
$ \aleph_0 < \# L$.  It follows from previous proofs that there are some languages
that are not recognized by a Turing machine. 
$\Box$

\vspace{.5in}

\begin{center}
\em ``There are more languages than there are Turing Machines.''
\end{center}
\es
\end{document}
%%%%%%%%%%%%%%%%%%%%%%%%%%% end of template1.tex %%%%%%%%%%%%%%%%%%%%%%%%%%%%%%%%

