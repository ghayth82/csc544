%template1.tex
%The following LaTeX source file represents the simplest kind of slide presentation; no overlays, no included graphics. Substitute your favorite style for ``pascal''. To create the PDF file template1.pdf, (1) be sure to use the prosper class, then (2) execute the command latex template1.tex, and (3) the command dvipdf template1.dvi.

%%%%%%%%%%%%%%%%%%%%%%%%%%%%%%% template1.tex %%%%%%%%%%%%%%%%%%%%%%%%%%%%%%%%%%%
\documentclass[a4paper,blends,pdf,colorBG,slideColor]{prosper}
% definitions for slides for CSC544
% Lutz Hamel, (c) 2007

\hypersetup{pdfpagemode=FullScreen}

\usepackage{amssymb}
\usepackage{latexsym}
\usepackage{amsmath}
%\usepackage[usenames]{color}
\usepackage{xypic}


\newcommand{\term}[1]{\ensuremath{\mbox{\bf #1}}}
\newcommand{\nonterm}[1]{\ensuremath{\mbox{#1}}}
\newcommand{\ifstmt}[3]{\ensuremath{{\bf if}\; {#1}\;{\bf then}\;{#2}\;{\bf else}\;{#3}\;\term{end}}}
\newcommand{\whilestmt}[2]{\ensuremath{{\bf while}\; {#1}\;{\bf do}\;{#2}\; \term{end}}}
\newcommand{\funcstmt}[3]{\ensuremath{{\bf fun}\; {#1}\; {\bf is}\; {#2} \; {\bf return}\; {#3}}}
\newcommand{\syntaxset}[1]{\ensuremath{\mbox{\bf #1}}}
\newcommand{\orbar}{\;|\;}
\newcommand{\bs}[1]{\begin{slide}{#1}\ptsize{8}}
\newcommand{\es}{\end{slide}}
\newcommand{\co}{\,\colon\;}
\newcommand{\pair}[2]{\ensuremath{\langle {#1}, {#2} \rangle}}
\newcommand{\encode}[1]{\ensuremath{\langle {#1} \rangle}}
\newcommand{\mytab}{\makebox[.15in]{}}
%\newcommand{\abs}[1]{{\mid{#1}\mid}}
\newcommand{\abs}[1]{{|{#1}|}}
\newcommand{\ol}[1]{\overline{#1}}

\newcommand{\qaccept}{\ensuremath{q_{\mbox{\tiny accept}}}}
\newcommand{\qreject}{\ensuremath{q_{\mbox{\tiny reject}}}}
\newcommand{\accept}{{\em accept}}
\newcommand{\reject}{{\em reject}}

\newcommand{\machine}[1]{
	\begin{quote}
	{#1}
	\end{quote}
	}

\newcommand{\fdef}[1]{
	\begin{center}
	\fbox{
	\begin{minipage}{3.5in}
	{\bf Definition:}
	{#1}
	\end{minipage}
	}
	\end{center}
	}

\newcommand{\ftheorem}[1]{
	\begin{center}
	\fbox{
	\begin{minipage}{3.5in}
	{\bf Theorem:}
	{#1}
	\end{minipage}
	}
	\end{center}
	}

\newcommand{\flemma}[1]{
	\begin{center}
	\fbox{
	\begin{minipage}{3.5in}
	{\bf Lemma:}
	{#1}
	\end{minipage}
	}
	\end{center}
	}


\newcommand{\fframe}[1]{
	\begin{center}
	\fbox{
	\begin{minipage}{3.5in}
	{#1}
	\end{minipage}
	}
	\end{center}
	}

\newcommand{\nframe}[1]{
	\begin{center}
	\begin{minipage}{3.5in}
	{#1}
	\end{minipage}
	\end{center}
	}

\begin{document}

\bs{The Mathematics of Functions}
The $\lambda$-calculus is very low level.  Here we investigate functions in a more abstract (mathematical) setting.

This is very similar to writing algorithms in English prose rather than writing actual machine code for the 
Turing machines.
\es

\bs{The Mathematics of Functions}
\small
Function Application and Composition:

Let $f\co A \rightarrow B$ be a (total) function from $A$ to $B$, then
for every value $x\in A$ we obtain a value $y\in B$,
\[
f x = y
\]
Function application is expressed by the {\em juxtaposition} of the function and its argument and
is evaluated from right to left.

Now assume that we have another function $g\co B \rightarrow C$ from $B$ into $C$, then
we can apply the function $g$ to the result of $f$.  For every value in $x\in A$
we obtain a value $z\in C$,
\[
g f x = g y = z
\]
In other words, we just constructed a new function, call it $h\co A \rightarrow C$, such that
\[
h x = g f x = g y = z
\]
We can express the same idea using function composition, $\circ$, without having to 
explicitly reference any values in $A$, $B$, or $C$,
\[
h = g \circ f
\]
and we say that ``$h$ is the composition of the function $g$ with function $f$''.

Note, that function composition is computed from right to left.

\es

\bs{The Mathematics of Functions}
\small
The Tuple:

Given two elements $x\in A$ and $y\in B$, then the tuple constructs an element of the
cross-product $A\times B$,
\[
x\in A, y\in B \Rightarrow (x,y)\in A\times B
\]

This is an important construction because it lets us apply functions to pairs (tuples) of values.
Assume we have a function $f\co A\times B \rightarrow C$.  This function
can only be applied to values in the cross-product $A\times B$, but we know how to construct
these values -- yes, the tuple,
\[
f {\color{red}(x,y)} = z
\]
for $x\in A$, $y\in B$, and $z\in C$.

Notice that we consider the tuple and function application separate computational steps.

We can of course generalize this to arbitrarily complex tuples, let $x_1\in X_1,\ldots,x_n\in X_n$
and let $f\co X_1\times\ldots\times X_n \rightarrow Y$, then
\[
f {\color{red}(x_1,\ldots,x_n)} = y
\]
for $y\in Y$.
\es


\bs{The Mathematics of Functions}
\small
Tuples of Functions:

Something interesting happens when we construct tuples of functions, let $f\co A \rightarrow B$
and let $g\co A \rightarrow B$, then
\[
(f,g) \in (A\rightarrow B)\times (A\rightarrow B)
\]
The pair of functions acts in parallel on an input in $A$ and produces a pair of output values in $B\times B$.

Let $a\in A$ and $b_f, b_g\in B$, then
\[
(f,g) a = (f a, g a) = (b_f, b_g)
\]
with $f a = b_f$ and $g a= b_g$.

Something a little bit more complicated.  Let $f\co X\times Y \rightarrow Z$ and $g\co X\times Y \rightarrow Z$ with $x\in X, y\in Y$ and $z_1, z_2\in Z$,
\[
(f,g)(x,y) = \left (f (x,y), g (x,y)\right) = \left (z_1, z_2\right )
\]
where $f (x,y) = z_1$ and $g (x,y) = z_2$.
\es

\bs{The Mathematics of Functions}
\small
Projection Functions:

Given a tuple
\[
(x_1,\ldots,x_n)
\]
with $x_1\in X_1,\ldots,x_n\in X_n$, then we can project the $i^{\mbox{\rm th}}$ component of the 
tuple with the projection function $p^{(n)}_i\co X_1\times\ldots\times X_n \rightarrow X_i$,
\[
p^{(n)}_i {\color{red}(x_1,\ldots,x_n)} = x_i
\]
with $1 \le i \le n$.

A more concrete example, let $x\in A$ and $y\in B$, then
\begin{align*}
p^{(2)}_1 (x,y) &= x\\
p^{(2)}_2 (x,y) &= y\\
\end{align*}
but
\begin{align*}
p^{({\color{red}3})}_1 (x,y) &= ???\\
p^{(2)}_{\color{red}5} (x,y) &= ???\\
\end{align*}
\es

\bs{The Mathematics of Functions}
\small
Putting Projection Functions, Tuples, and Composition together:

Let $f\co X_1\times X_2 \rightarrow Y$ and let $(x_1,x_2,x_3)\in X_1\times X_2\times X_3$, then
\[
f \circ (p_1^{(3)},p_2^{(3)}) \co X_1\times X_2 \times X_3 \rightarrow Y
\]
Applying this function to our tuple we have
\[
f (p_1^{(3)},p_2^{(3)})(x_1,x_2,x_3) = f (p_1^{(3)}(x_1,x_2,x_3),p_2^{(3)}(x_1,x_2,x_3)) = f (x_1,x_2)
\]

Here is another example, let $(x_1,x_2)\in X_1\times X_2$, then
\[
(p_1^{(2)},p_2^{(2)}) (x_1,x_2) = (p_1^{(2)}(x_1,x_2),p_2^{(2)}(x_1,x_2)) = (x_1,x_2)
\]
\es



\end{document}
%%%%%%%%%%%%%%%%%%%%%%%%%%% end of template1.tex %%%%%%%%%%%%%%%%%%%%%%%%%%%%%%%%

