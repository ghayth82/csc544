%template1.tex
%The following LaTeX source file represents the simplest kind of slide presentation; no overlays, no included graphics. Substitute your favorite style for ``pascal''. To create the PDF file template1.pdf, (1) be sure to use the prosper class, then (2) execute the command latex template1.tex, and (3) the command dvipdf template1.dvi.

%%%%%%%%%%%%%%%%%%%%%%%%%%%%%%% template1.tex %%%%%%%%%%%%%%%%%%%%%%%%%%%%%%%%%%%
\documentclass[a4paper,blends,pdf,colorBG,slideColor]{prosper}
% definitions for slides for CSC544
% Lutz Hamel, (c) 2007

\hypersetup{pdfpagemode=FullScreen}

\usepackage{amssymb}
\usepackage{latexsym}
\usepackage{amsmath}
%\usepackage[usenames]{color}
\usepackage{xypic}


\newcommand{\term}[1]{\ensuremath{\mbox{\bf #1}}}
\newcommand{\nonterm}[1]{\ensuremath{\mbox{#1}}}
\newcommand{\ifstmt}[3]{\ensuremath{{\bf if}\; {#1}\;{\bf then}\;{#2}\;{\bf else}\;{#3}\;\term{end}}}
\newcommand{\whilestmt}[2]{\ensuremath{{\bf while}\; {#1}\;{\bf do}\;{#2}\; \term{end}}}
\newcommand{\funcstmt}[3]{\ensuremath{{\bf fun}\; {#1}\; {\bf is}\; {#2} \; {\bf return}\; {#3}}}
\newcommand{\syntaxset}[1]{\ensuremath{\mbox{\bf #1}}}
\newcommand{\orbar}{\;|\;}
\newcommand{\bs}[1]{\begin{slide}{#1}\ptsize{8}}
\newcommand{\es}{\end{slide}}
\newcommand{\co}{\,\colon\;}
\newcommand{\pair}[2]{\ensuremath{\langle {#1}, {#2} \rangle}}
\newcommand{\encode}[1]{\ensuremath{\langle {#1} \rangle}}
\newcommand{\mytab}{\makebox[.15in]{}}
%\newcommand{\abs}[1]{{\mid{#1}\mid}}
\newcommand{\abs}[1]{{|{#1}|}}
\newcommand{\ol}[1]{\overline{#1}}

\newcommand{\qaccept}{\ensuremath{q_{\mbox{\tiny accept}}}}
\newcommand{\qreject}{\ensuremath{q_{\mbox{\tiny reject}}}}
\newcommand{\accept}{{\em accept}}
\newcommand{\reject}{{\em reject}}

\newcommand{\machine}[1]{
	\begin{quote}
	{#1}
	\end{quote}
	}

\newcommand{\fdef}[1]{
	\begin{center}
	\fbox{
	\begin{minipage}{3.5in}
	{\bf Definition:}
	{#1}
	\end{minipage}
	}
	\end{center}
	}

\newcommand{\ftheorem}[1]{
	\begin{center}
	\fbox{
	\begin{minipage}{3.5in}
	{\bf Theorem:}
	{#1}
	\end{minipage}
	}
	\end{center}
	}

\newcommand{\flemma}[1]{
	\begin{center}
	\fbox{
	\begin{minipage}{3.5in}
	{\bf Lemma:}
	{#1}
	\end{minipage}
	}
	\end{center}
	}


\newcommand{\fframe}[1]{
	\begin{center}
	\fbox{
	\begin{minipage}{3.5in}
	{#1}
	\end{minipage}
	}
	\end{center}
	}

\newcommand{\nframe}[1]{
	\begin{center}
	\begin{minipage}{3.5in}
	{#1}
	\end{minipage}
	\end{center}
	}

\begin{document}

\bs{First Order Logic}
A logic is a formal language with strings over the alphabet $\Sigma$:
\[
\Sigma = \{ \wedge, \vee, \neg, (, ), \forall, \exists, x, R_1, \ldots, R_k \}
\]

A well-formed formula (WFF) is a string over this alphabet and can be defined inductively
as follows.  A string $\phi$ is a WFF if it
\begin{itemize}
\item is an atomic formula of the form $R_i(x_1,\ldots,x_n)$ with $R_i$ called a relation symbol, or
\item has one of the following forms: $\phi_1 \wedge \phi_2$, $\phi_1 \vee \phi_2$, $\neg \phi_1$,
where $\phi_1$ and $\phi_2$ are smaller formulas, or
\item has the form $\exists x_i[\phi_1]$ or $\forall x_i[\phi_1]$, where $\phi_1$ is a smaller formula.
\end{itemize}

{\bf Notes:} Quantifiers bind variables within their scope (square brackets).  If a variable is not
bound in a formula then we call it a {\bf \em free variable}.  Formulas with no free variables are
called {\bf\em sentences} or {\bf\em statements}.  Also it is not necessary to specify the 
implication operator as part of the alphabet because $(p \rightarrow q) \equiv (\neg p \vee q)$
\es

\bs{First Order Logic}
WFFs:
\begin{itemize}
\item $R_1(x_1) \wedge R_2(x_1,x_2,x_3)$
\item $\forall x_1[R_1(x_1) \wedge R(x_1,x_2,x_3)]$
\item$\forall x_1\exists x_2 \exists x_3 [R_1(x_1) \wedge R_2(x_1,x_2,x_3)]$
\end{itemize}

{\bf Observation:} Only the last WFF above is a sentence.

We say, {\em for all $x_1$ there exist $x_2$ and $x_3$ such that $R_1(x_1)$ and 
$R_2(x_1,x_2,x_3)$ are true.}
\es

\bs{First Order Logic}
We set this system up so that we can reason about sentences such as,
\begin{enumerate}
\item $\forall a,b,c,n[(a,b,c > 0 \wedge n > 2) \rightarrow a^n + b^n \ne c^n].$ (Fermat's Last Theorem)
\item $\forall n \exists a,b,c,d[(a,b,c,d,n \ge 0) \rightarrow a^2 + b^2 + c^2 + d^2 = n].$ (Lagrange's Four-Square Theorem)
\end{enumerate}
First order logic  is a language rich enough to
formalize mathematics, but as we will see, it has its limitations (c.f. {\em Principia Mathematica}, Whitehead and Russell, 1913).  

And it is rich enough to serve as a Turing-complete programming language.
\es

\bs{First Order Logic}
In order for a logic to make sense we need to assign it meaning.  We do this by assigning
the syntax to specific mathematical constructs, called the {\bf\em model}.

A model is made up of a {\bf\em universe} and a set of relations, one for each relation symbol
in the logic.

{\bf Example:} Let our $\Sigma$ be
\[
\Sigma = \{ \wedge, \vee, \neg, (, ), \forall, \exists, x, R_1(\cdot,\cdot) \},
\]
then a model for this logic is $M_1 = ({\mathbb N}, \le)$, with $x \mapsto {\mathbb N}$ and $R_1 \mapsto \le$.
Here, $\mathbb N$ is the universe and the relation $\le\in {\mathbb N} \times {\mathbb N}$ is the {\bf \em interpretation} of the binary relation symbol $R_1$. $\Box$

{\bf NOTE:}  It is not customary to build an explicit interpretation for the logical operators of the language,
although we could, but this would just clutter our notation since those interpretation will not change.
An explicit model of the logical operators is the universe of boolean values with the standard boolean relations/functions assigned to 
the boolean operators in the logic.

\es

\bs{First Order Logic}
Given a logic and let $M$ be a model, we then write $Th(M)$ for the (usually infinite) collection of all sentences in the logic that are true in $M$.  We call $Th(M)$ the {\em theory of $M$}.  

If some sentence $s$ is true in $M$ then we say that $M$ {\em satisfies} the sentence $s$ and we write
\[
M \models s
\]
Observe that
\[
M \models s \mbox{ iff } s \in Th(M)
\]
Therefore we often write $Th(M) \models s$ instead of $M\models s$.
\es

\bs{First Order Logic}
{\bf Example:} Given our logic $\Sigma = \{ \wedge, \vee, \neg, (, ), \forall, \exists, x, R_1(\cdot,\cdot) \}$
with the model $M_1 = ({\mathbb N}, {\color{red}\le})$ with the obvious interpretation, it is easy to show that
\[
({\mathbb N}, \le) \models \forall x \forall y[R_1(x,y) \vee R_1(y,x)].
\]
It is clearly satisfied, since for any assignment $x \mapsto a$ and $y \mapsto b$ for $a,b \in {\mathbb N}$
we have $a \le b \mbox{ or } b \le a$. $\Box$

{\bf Example:} Given our logic $\Sigma = \{ \wedge, \vee, \neg, (, ), \forall, \exists, x, R_1(\cdot,\cdot) \}$
with the model $M_2 = ({\mathbb N}, {\color{red}<})$ with the obvious interpretation, we show that
sentence
\[
({\mathbb N}, <) \not\models \forall x \forall y[R_1(x,y) \vee R_1(y,x)].
\]
In this case the sentence is {\bf not satisfied} because for the assignment $x \mapsto a$ and $y \mapsto a$ for $a \in {\mathbb N}$
we have $a < a \mbox{ or } a < a$. $\Box$
\es

\bs{First Order Logic}
{\bf Example:} Given a logic $\Sigma = \{ \wedge, \vee, \neg, (, ), \forall, \exists, x, R_1(\cdot,\cdot,\cdot) \}$
with the model $M_3 = ({\mathbb R}, +)$ with the obvious interpretation, we can show that
sentence
\[
({\mathbb R}, +) \models \forall y \exists x[R_1(x,x,y)].
\]
Here, the sentence is  satisfied because for any assignments $x \mapsto a$ and $y \mapsto b$ for $a,b \in {\mathbb R}$
we have $+(a,a,b)$ or in standard mathematical notation $b = a + a$. 
Note, that if we had chosen a model $({\mathbb N}, +)$ the sentence would be not be satisfied. $\Box$
\es

\bs{A Decidable Theory}
\fframe{\begin{center}{\bf Theorem:} The theory $Th({\mathbb N}, +)$ is decidable.\end{center}}

{\bf Observations:} What does it mean for a theory to be decidable? It means that we can
decide whether a particular sentence belongs to the theory or not.  That is, we can treat the theory
$Th({\mathbb N}, +)$ as a language and we can construct a decider for this language. We can easily show that
the following holds,
\[
Th({\mathbb N}, +) \models \forall x \exists y[y = x + x].
\]
However, switching the quantifiers gives us,
\[
Th({\mathbb N}, +) \not \models \exists x \forall y[y = x + x].
\]
Therefore this sentence is not a member of the theory.
The book shows decidability by constructing an FA that decides the language.

\es

\bs{An Undecidable Theory}

The following theorem has deep philosophical consequences.  It shows that mathematics cannot
be mechanized.  It also shows that certain problems in number theory are not algorithmic, which
was a big surprise to mathematicians in the early 1900's.  It was then believed that all problems in
mathematics can be solved algorithmically, one just had to come up with a clever algorithm.

\fframe{\begin{center}{\bf Theorem:} The theory $Th({\mathbb N}, +, \times)$ is undecidable.\end{center}}

{\bf Observations:} This means, there is no algorithm that will halt on all sentences $\phi$ over
the appropriate alphabet.  
What is perhaps most surprising is the simple structure of this undecidable logic.
This means that there are fundamental algorithmic limitations in mathematics.
The proof is by reduction of $A_{TM}$ to $Th({\mathbb N}, +, \times)$.

\es

\bs{Proofs}
Given a theory $Th(M)$, a proof $\pi$ of a statement $\phi$ is a sequence of statements
\[
S_1,S_2,\ldots,S_l
\]
such that each $S_i \in  Th(M)$ and $S_l = \phi$.  Furthermore, each $S_{i+1}$ must follow from $S_{i}$ by an application of an inference rule\footnote{\tiny For inference rules for first-order logic see http://en.wikipedia.org/wiki/List\_of\_rules\_of\_inference} of that logic.   

If $\phi$ is provable, i.e. it has a proof, then we write
\[
Th(M) \vdash \phi
\]
If the logic is {\em sound} then
\[
Th(M) \vdash \phi \Rightarrow M \models \phi
\]
If the logic is {\em complete} then
\[
M \models \phi \Rightarrow Th(M) \vdash \phi
\]

\es

\bs{Sound and Complete Logic}
\fframe{
We call a logic sound and complete if for some model $M$ we have
\[
Th(M) \vdash \phi \Leftrightarrow M \models \phi
\]
}
\es
\bs{Incompleteness Theorem}
For the following we assume that proof systems have the following properties,
\begin{enumerate}
\item The correctness of a proof of a statement can be checked by machine.  Formally,
\[
\{\langle\phi,\pi\rangle | \text{$\pi$ is a proof of $\phi$}\}
\]
is decidable.
\item The system of proofs is {\em sound}.  
\end{enumerate}
\es

\bs{Incompleteness Theorem}
{\bf Theorem:} The collection of provable statements in $Th({\mathbb N}, +, \times)$ is
Turing-recognizable.

{\bf Proof:} The following algorithm $P$ accepts its input $\phi$ if $\phi$ is provable.
The algorithm $P$ tries out every possible string as a candidate for a proof $\pi$ of $\phi$ using
the proof checker assumed in the previous slide.  If it finds a string that is a proof, it accepts.
$\Box$
\es

\bs{Incompleteness Theorem}
Kurt G\"odels famous incompleteness theorem:

\begin{center}
\fbox{{\bf Theorem:} Some true statements in $Th({\mathbb N}, +, \times)$ are not provable.}
\end{center}

{\bf Proof:} Proof by contradiction. We assume that all statements are provable.  We show
that with this assumption we can construct a machine $D$ that can decide $Th({\mathbb N}, +, \times)$;
a contradiction to our earlier results. We conclude that not all statements are provable.

On input $\phi$ the algorithm $D$ operates by running algorithm $P$ in parallel on inputs
$\phi$ and $\neg \phi$.
One of these two statements is true and thus by our assumption is provable.
Therefore $P$ must halt on one of the two inputs.
By our assumptions on proof systems, if $\phi$ is provable, then $\phi$ is true and if 
$\neg \phi$ is true then $\phi$ is false. So algorithm $D$ can decide the truth or
falsity of $\phi$, that is, $D$ is a decider for $Th({\mathbb N}, +, \times)$.
This is a contradiction, therefore our assumption must be false. $\Box$
\es


\end{document}
%%%%%%%%%%%%%%%%%%%%%%%%%%% end of template1.tex %%%%%%%%%%%%%%%%%%%%%%%%%%%%%%%%

