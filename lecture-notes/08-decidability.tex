%template1.tex
%The following LaTeX source file represents the simplest kind of slide presentation; no overlays, no included graphics. Substitute your favorite style for ``pascal''. To create the PDF file template1.pdf, (1) be sure to use the prosper class, then (2) execute the command latex template1.tex, and (3) the command dvipdf template1.dvi.

%%%%%%%%%%%%%%%%%%%%%%%%%%%%%%% template1.tex %%%%%%%%%%%%%%%%%%%%%%%%%%%%%%%%%%%
\documentclass[a4paper,blends,pdf,colorBG,slideColor]{prosper}
% definitions for slides for CSC544
% Lutz Hamel, (c) 2007

\hypersetup{pdfpagemode=FullScreen}

\usepackage{amssymb}
\usepackage{latexsym}
\usepackage{amsmath}
%\usepackage[usenames]{color}
\usepackage{xypic}


\newcommand{\term}[1]{\ensuremath{\mbox{\bf #1}}}
\newcommand{\nonterm}[1]{\ensuremath{\mbox{#1}}}
\newcommand{\ifstmt}[3]{\ensuremath{{\bf if}\; {#1}\;{\bf then}\;{#2}\;{\bf else}\;{#3}\;\term{end}}}
\newcommand{\whilestmt}[2]{\ensuremath{{\bf while}\; {#1}\;{\bf do}\;{#2}\; \term{end}}}
\newcommand{\funcstmt}[3]{\ensuremath{{\bf fun}\; {#1}\; {\bf is}\; {#2} \; {\bf return}\; {#3}}}
\newcommand{\syntaxset}[1]{\ensuremath{\mbox{\bf #1}}}
\newcommand{\orbar}{\;|\;}
\newcommand{\bs}[1]{\begin{slide}{#1}\ptsize{8}}
\newcommand{\es}{\end{slide}}
\newcommand{\co}{\,\colon\;}
\newcommand{\pair}[2]{\ensuremath{\langle {#1}, {#2} \rangle}}
\newcommand{\encode}[1]{\ensuremath{\langle {#1} \rangle}}
\newcommand{\mytab}{\makebox[.15in]{}}
%\newcommand{\abs}[1]{{\mid{#1}\mid}}
\newcommand{\abs}[1]{{|{#1}|}}
\newcommand{\ol}[1]{\overline{#1}}

\newcommand{\qaccept}{\ensuremath{q_{\mbox{\tiny accept}}}}
\newcommand{\qreject}{\ensuremath{q_{\mbox{\tiny reject}}}}
\newcommand{\accept}{{\em accept}}
\newcommand{\reject}{{\em reject}}

\newcommand{\machine}[1]{
	\begin{quote}
	{#1}
	\end{quote}
	}

\newcommand{\fdef}[1]{
	\begin{center}
	\fbox{
	\begin{minipage}{3.5in}
	{\bf Definition:}
	{#1}
	\end{minipage}
	}
	\end{center}
	}

\newcommand{\ftheorem}[1]{
	\begin{center}
	\fbox{
	\begin{minipage}{3.5in}
	{\bf Theorem:}
	{#1}
	\end{minipage}
	}
	\end{center}
	}

\newcommand{\flemma}[1]{
	\begin{center}
	\fbox{
	\begin{minipage}{3.5in}
	{\bf Lemma:}
	{#1}
	\end{minipage}
	}
	\end{center}
	}


\newcommand{\fframe}[1]{
	\begin{center}
	\fbox{
	\begin{minipage}{3.5in}
	{#1}
	\end{minipage}
	}
	\end{center}
	}

\newcommand{\nframe}[1]{
	\begin{center}
	\begin{minipage}{3.5in}
	{#1}
	\end{minipage}
	\end{center}
	}

\begin{document}

\bs{Proofs using Deciders}

{\bf Proposition:} Let $L_1$ and $L_2$ be decidable languages, then the concatenation
$L = L_1 \circ L_2$ is also decidable.

{\bf Proof:} We show decidability of $L$ by constructing a decider for it.
Let $M_1$ and $M_2$ be deciders for $L_1$ and $L_2$, respectively, then
we can construct a decider $M$ for $L$ as follows:

\machine{
M = "On input $w$,
\begin{enumerate}
\item For each way to split $w$ into two parts, $w = w_1 w_2$, do:
\item \mytab Run $M_1$ on $w_1$.
\item \mytab Run $M_2$ on $w_2$.
\item If for any combination $M_1$ and $M_2$ accept, \accept; otherwise, \reject."
\end{enumerate}
}
\es

\bs{Decidability}
Let us study some "standard" machines that are deciders.  

It turns out that these standard machines help us construct more complicated proofs.
\es

\bs{Decidability}
{\small
\fframe{
{\bf Theorem:} The language \[A_{\mbox{\tiny DFA}} = \{ \encode{B,w} | \mbox{$B$ is a DFA
that accepts string $w$}\}\] is decidable.
}

{\bf Proof:} We construct a decider $M_{\mbox{\tiny DFA}}$ for  $A_{\mbox{\tiny DFA}}$.

\machine{
$M_{\mbox{\tiny DFA}}$ = "On input \encode{B,w}, where $B$ is a DFA and $w$ is a string:
\begin{enumerate}
\item Simulate $B$ on input $w$.
\item If the simulation ends in an accept state of the DFA, \accept; otherwise, \reject."
\end{enumerate}
}
}
\es

\bs{Decidability}
{\small
\fframe{
{\bf Theorem:} The language \[A_{\mbox{\tiny NFA}} = \{ \encode{B,w} | \mbox{$B$ is an NFA
that accepts string $w$}\}\] is decidable.
}

{\bf Proof:} We construct a decider $M_{\mbox{\tiny NFA}}$ for  $A_{\mbox{\tiny NFA}}$.

\machine{
$M_{\mbox{\tiny NFA}}$ = "On input \encode{B,w}, where $B$ is an NFA and $w$ is a string:
\begin{enumerate}
\item Convert NFA $B$ into an equivalent DFA $B'$ (this is algorithmic, so a TM can do it).
\item Run $M_{\mbox{\tiny DFA}}$ on input \encode{B',w}.
\item If $M_{\mbox{\tiny DFA}}$ accepts, \accept; otherwise, \reject."
\end{enumerate}
}
}
\es

\bs{Decidability}
{\small
\fframe{
{\bf Theorem:} The language \[A_{\mbox{\tiny REX}} = \{ \encode{R,w} | \mbox{$R$ is a regular expression that generates
string $w$}\}\] is decidable.
}

{\bf Proof:} We construct a decider $M_{\mbox{\tiny REX}}$ for  $A_{\mbox{\tiny REX}}$.

\machine{
$M_{\mbox{\tiny REX}}$ = "On input \encode{R,w}, where $R$ is a regular expression and $w$ is a string:
\begin{enumerate}
\item Convert regular expression $R$ into an equivalent DFA $R'$ (this is algorithmic, so a TM can do it).
\item Run $M_{\mbox{\tiny DFA}}$ on input \encode{R',w}.
\item If $M_{\mbox{\tiny DFA}}$ accepts, \accept; otherwise, \reject."
\end{enumerate}
}
}
\es


\bs{Decidability}
{\small
\fframe{
{\bf Theorem:} The language \[E_{\mbox{\tiny DFA}} = \{ \encode{A} | \mbox{$A$ is a DFA and $L(A) = \emptyset$}\}\] is decidable.
}

{\bf Proof:} We construct a decider $M_{E_{\mbox{\tiny DFA}}}$ for  $E_{\mbox{\tiny DFA}}$.

\machine{
$M_{E_{\mbox{\tiny DFA}}}$ = "On input \encode{A}, where $A$ is a DFA:
\begin{enumerate}
\item Mark the start state of $A$.
\item Repeat until no new states get marked:
\item \mytab Mark any state that has a transition coming into it from  any state
already marked.
\item If no accept state is marked, \accept; otherwise, \reject."
\end{enumerate}
}
}
\es

\bs{Decidability}
{\small
\fframe{
{\bf Theorem:} The language \[EQ_{\mbox{\tiny DFA}} = \{ \encode{A,B} | \mbox{$A$ and $B$ are DFAs and $L(A) = L(B)$}\}\] is decidable.
}

{\bf Proof:}  In order to show this we construct a new machine $C$ that accepts only
those string that either $A$ or $B$ accepts but not both, i.e.
\begin{equation}
\label{eq:symmetric-diff}
L(C) = (L(A) \cap \overline{L(B)}) \cup (\overline{L(A)} \cap L(B)).
\end{equation}
Furthermore, we require that $L(C) = \emptyset$ which implies that $L(A) = L(B)$ as
needed.

Regular languages are closed under complementation, union, and intersection; therefore
we are able to construct machine $C$ according to (\ref{eq:symmetric-diff}).

We can now construct a decider $M_{EQ_{\mbox{\tiny DFA}}}$ for  $EQ_{\mbox{\tiny DFA}}$.

\machine{
$M_{EQ_{\mbox{\tiny DFA}}}$ = "On input \encode{A,B}, where $A$ $B$ are DFAs:
\begin{enumerate}
\item Construct DFA $C$ as described in (\ref{eq:symmetric-diff}).
\item Run $M_{E_{\mbox{\tiny DFA}}}$ on input \encode{C}.
\item If $M_{E_{\mbox{\tiny DFA}}}$ accepts, \accept; otherwise, \reject."
\end{enumerate}
}
}
\es



\bs{Decidability}
{\small
\fframe{
{\bf Theorem:} The language \[A_{\mbox{\tiny CFG}} = \{ \encode{G,w} | \mbox{$G$ is a CFG that generates
string $w$}\}\] is decidable.
}

{\bf Proof Attempt 1:} We construct a decider $M_{\mbox{\tiny CFG}}$ for  $A_{\mbox{\tiny CFG}}$.

\machine{
$M_{\mbox{\tiny CFG}}$ = "On input \encode{G,w}, where $G$ is a CFG and $w$ is a string:
\begin{enumerate}
\item Convert CFG $G$ into an equivalent PDA $G'$ (this is algorithmic, so a TM can do it).
\item Simulate $G'$ on input $w$.
\item If the simulation ends in an accept state, \accept; otherwise, \reject."
\end{enumerate}
}

$\Rightarrow$ Turns out that this does {\bf\em not} work because a PDA is a non-deterministic machine
and can have branches could go on computing forever.

{\bf NOTE:} in practice we actually do use PDAs to decide membership, since infinite
computations are not a problem since we always "guess" just the appropriate transitions
using dynamic programming techniques and when we cannot guess the appropriate transition that usually signals an error.
}
\es

\bs{Decidability}
{\small
\fframe{
{\bf Theorem:} The language \[A_{\mbox{\tiny CFG}} = \{ \encode{G,w} | \mbox{$G$ is a CFG that generates
string $w$}\}\] is decidable.
}

{\bf Proof Attempt 2:} We construct a decider $M_{\mbox{\tiny CFG}}$ for  $A_{\mbox{\tiny CFG}}$.

\machine{
$M_{\mbox{\tiny CFG}}$ = "On input \encode{G,w}, where $G$ is a CFG and $w$ is a string:
\begin{enumerate}
\item Convert CFG $G$ into an equivalent Chomsky normal form $G'$ (this is algorithmic, so a TM can do it).
\item List all derivations with $2n - 1$ steps, where $n$ is the length of $w$,
except if $n = 0$, then list all derivations with 1 step.
\item If any of these derivations generate $w$, \accept; otherwise, \reject."
\end{enumerate}
}

{\bf NOTE:} now the computation is bounded by $2n-1$ $\rightarrow$ decider.
}
\es


\bs{Decidability}
{\small
\fframe{
{\bf Theorem:} The language \[E_{\mbox{\tiny CFG}} = \{ \encode{G} | \mbox{$G$ is a CFG and $L(G) = \emptyset$}\}\] is decidable.
}

{\bf Proof:} We construct a decider $M_{E_{\mbox{\tiny CFG}}}$ for  $E_{\mbox{\tiny CFG}}$.

\machine{
$M_{E_{\mbox{\tiny CFG}}}$ = "On input \encode{G}, where $G$ is a CFG:
\begin{enumerate}
\item Mark all terminal symbols in $G$.
\item Repeat until no new variables get marked:
\item \mytab Mark any variable $A$ where $G$ has a rule $A \rightarrow U_1 U_2\ldots U_k$ and each symbol $U_1 U_2\ldots U_k$ has already been marked.
\item If the start symbol is not marked, \accept; otherwise, \reject."$\Box$
\end{enumerate}
}
}

{\tiny
{\bf Example:}
\begin{eqnarray*}
S &\rightarrow& ASA\\
S &\rightarrow& a B\\
A &\rightarrow& B\\
A & \rightarrow & S\\
B & \rightarrow& b\\
B &\rightarrow& \epsilon
\end{eqnarray*}
}
\es


\bs{Decidability}
{\small
\fframe{
{\bf Theorem:} Every context-free language is decidable.
}

{\bf Proof:} Let $A$ be a CFL, then we know that there must be a CFG $G$ such that
$L(G) = A$.  Now we can construct a decider $M_A$ for $A$ as follows:

\machine{
$M_A$ = "On input $w$, where $w$ is some string:
\begin{enumerate}
\item Run $M_{\mbox{\tiny CFG}}$ on \encode{G,w}.
\item If $M_{\mbox{\tiny CFG}}$ accepts, \accept; otherwise, \reject."
\end{enumerate}
}
}
\es

\bs{Example}

{\bf Example:} Let \[A = \{ \encode{M} | \mbox{$M$ is a DFA which does not accept any
string containing an odd number of $1$'s}\}.\]
Show that $A$ is decidable.

{\bf Proof:} We need to construct a decider $M$ for $A$.

\machine{
$M$ = "On input \encode{M}, where $M$ is a  DFA:
\begin{enumerate}
\item Construct a DFA $O$ that accepts every string containing an odd number of $1$'s.
\item Construct a DFA $B$ such that $L(B) = L(M) \cap L(O)$.
\item Run $M_{E_{\mbox{\tiny DFA}}}$ on input \encode{B}.
\item If $M_{E_{\mbox{\tiny DFA}}}$ accepts, \accept; otherwise, \reject."
\end{enumerate}
}
\es


\bs{The Halting Problem}
{\small
\fframe{{\bf Theorem:} The language
\[
A_{TM} = \{ \encode{M,w} | \mbox{$M$ is a TM and $M$ accepts $w$.}\}
\]
is undecidable.
}
{\bf Proof:} By contradiction.  Assume that there exists a decider $H$ for language $A_{TM}$. Then,
\[
H(\encode{M,w}) = \left \{
\begin{array}{ll}
\mbox{\accept} & \mbox{if $M$ accepts $w$}\\
\mbox{\reject} & \mbox{if $M$ does not accept $w$}
\end{array}
\right .
\]
We construct a new TM $D$ such that
\begin{quote}
$D$ = "On input $\encode{Q}$ where $Q$ is a TM:
\begin{enumerate}
\item Run $H$ on input $\encode{Q, \encode{Q}}$.
\item Output the opposite of what $H$ outputs; that is, if $H$ accepts, {\em reject}
and if 
$H$ rejects, \accept."
\end{enumerate}
\end{quote}
Now,
\[
D(\encode{M}) = \left \{
\begin{array}{ll}
\mbox{\accept} & \mbox{if $M$ does not accept $\encode{M}$}\\
\mbox{\reject} & \mbox{if $M$ accepts $\encode{M}$}
\end{array}
\right .
\]
But...
}
\es

\bs{The Halting Problem}
{\small
Now consider,
\[
D({\color{red}\encode{D}}) = \left \{
\begin{array}{ll}
\mbox{\accept} & \mbox{if $D$ does not accept $\encode{D}$}\\
\mbox{\reject} & \mbox{if $D$ accepts $\encode{D}$}
\end{array}
\right .
\]
This is a contradiction.  Therefore, neither $H$ nor $D$ can exist and $A_{TM}$ is
undecidable.$\Box$
}
\es

\bs{$A_{TM}$ is Turing-Recognizable}
\fframe{{\bf Theorem:} The language
\[
A_{TM} = \{ \encode{M,w} | \mbox{$M$ is a TM and $M$ accepts $w$.}\}
\]
is Turing-Recognizable.
}
{\bf Proof:} We construct the TM $U$ that recognizes $A_{TM}$,
\begin{quote}
$U$ = "On input $\encode{M,w}$ where $M$ is a TM and $w$ is a string:
\begin{enumerate}
\item Simulate $M$ on string $w$.
\item If $M$ ever enters its accept state, \accept; if $M$ ever enters its reject state, \reject."
\end{enumerate}
\end{quote}
\es


\bs{\large Non-Turing-Recognizable Languages}
The undecidability  of $A_{TM}$ has profound ramifications $\Rightarrow$ the existence of languages that are non-algorithmic, that is, languages that are not Turing-recognizable.


One such language is the complement of $A_{TM}$.
\es

\bs{\Large Non-Turing-Recognizable Languages}
\fframe{{\bf Theorem:} The language $\overline{A_{TM}}$ is not Turing-recognizable.}

{\bf Proof:} By contradiction. Observe that $A_{TM}$ is undecidable.  In addition, observe
that $A_{TM}$ is Turing-recognizable.  Now, assume that $\overline{A_{TM}}$ is also
Turing-recognizable.  Observing that a string $w$ is either an element of $A_{TM}$
or an element of $\overline{A_{TM}}$ we can construct the following decider for
$A_{TM}$.
Let $M_1$ and $M_2$ be recognizers for $A_{TM}$ and $\overline{A_{TM}}$, respectively:

\begin{quote}
$M$ = "On input $w$, where $w$ is a string:
\begin{enumerate}
\item Run $M_1$ and $M_2$ in parallel on $w$.
\item If $M_1$ accepts, \accept; if $M_2$ accepts, \reject."
\end{enumerate}
\end{quote}

Note that this machine is a decider because it will halt on every input $w$.  
Also note, that this decider contradicts our theorem that $A_{TM}$ is 
undecidable.  Therefore, our assumption that $\overline{A_{TM}}$
is Turing-recognizable must be wrong.

This shows that $\overline{A_{TM}}$ is not Turing-recognizable.$\Box$
\es

\end{document}
%%%%%%%%%%%%%%%%%%%%%%%%%%% end of template1.tex %%%%%%%%%%%%%%%%%%%%%%%%%%%%%%%%

