%template1.tex
%The following LaTeX source file represents the simplest kind of slide presentation; no overlays, no included graphics. Substitute your favorite style for ``pascal''. To create the PDF file template1.pdf, (1) be sure to use the prosper class, then (2) execute the command latex template1.tex, and (3) the command dvipdf template1.dvi.

%%%%%%%%%%%%%%%%%%%%%%%%%%%%%%% template1.tex %%%%%%%%%%%%%%%%%%%%%%%%%%%%%%%%%%%
\documentclass[a4paper,blends,pdf,colorBG,slideColor]{prosper}
% definitions for slides for CSC544
% Lutz Hamel, (c) 2007

\hypersetup{pdfpagemode=FullScreen}

\usepackage{amssymb}
\usepackage{latexsym}
\usepackage{amsmath}
%\usepackage[usenames]{color}
\usepackage{xypic}


\newcommand{\term}[1]{\ensuremath{\mbox{\bf #1}}}
\newcommand{\nonterm}[1]{\ensuremath{\mbox{#1}}}
\newcommand{\ifstmt}[3]{\ensuremath{{\bf if}\; {#1}\;{\bf then}\;{#2}\;{\bf else}\;{#3}\;\term{end}}}
\newcommand{\whilestmt}[2]{\ensuremath{{\bf while}\; {#1}\;{\bf do}\;{#2}\; \term{end}}}
\newcommand{\funcstmt}[3]{\ensuremath{{\bf fun}\; {#1}\; {\bf is}\; {#2} \; {\bf return}\; {#3}}}
\newcommand{\syntaxset}[1]{\ensuremath{\mbox{\bf #1}}}
\newcommand{\orbar}{\;|\;}
\newcommand{\bs}[1]{\begin{slide}{#1}\ptsize{8}}
\newcommand{\es}{\end{slide}}
\newcommand{\co}{\,\colon\;}
\newcommand{\pair}[2]{\ensuremath{\langle {#1}, {#2} \rangle}}
\newcommand{\encode}[1]{\ensuremath{\langle {#1} \rangle}}
\newcommand{\mytab}{\makebox[.15in]{}}
%\newcommand{\abs}[1]{{\mid{#1}\mid}}
\newcommand{\abs}[1]{{|{#1}|}}
\newcommand{\ol}[1]{\overline{#1}}

\newcommand{\qaccept}{\ensuremath{q_{\mbox{\tiny accept}}}}
\newcommand{\qreject}{\ensuremath{q_{\mbox{\tiny reject}}}}
\newcommand{\accept}{{\em accept}}
\newcommand{\reject}{{\em reject}}

\newcommand{\machine}[1]{
	\begin{quote}
	{#1}
	\end{quote}
	}

\newcommand{\fdef}[1]{
	\begin{center}
	\fbox{
	\begin{minipage}{3.5in}
	{\bf Definition:}
	{#1}
	\end{minipage}
	}
	\end{center}
	}

\newcommand{\ftheorem}[1]{
	\begin{center}
	\fbox{
	\begin{minipage}{3.5in}
	{\bf Theorem:}
	{#1}
	\end{minipage}
	}
	\end{center}
	}

\newcommand{\flemma}[1]{
	\begin{center}
	\fbox{
	\begin{minipage}{3.5in}
	{\bf Lemma:}
	{#1}
	\end{minipage}
	}
	\end{center}
	}


\newcommand{\fframe}[1]{
	\begin{center}
	\fbox{
	\begin{minipage}{3.5in}
	{#1}
	\end{minipage}
	}
	\end{center}
	}

\newcommand{\nframe}[1]{
	\begin{center}
	\begin{minipage}{3.5in}
	{#1}
	\end{minipage}
	\end{center}
	}

\begin{document}

\bs{Reducibility}
We say that a problem $Q$ reduces to problem $P$ if we can use $P$ to solve $Q$.

In the context of decidability we have the following ``templates'':

\fframe{
\begin{equation}
\label{reduce-1}
\mbox{If $A$ reduces to $B$ and $B$ is decidable, then so is $A$.}
\end{equation}
}
and
\fframe{
\begin{equation}
\label{reduce-2}
\mbox{If $A$ reduces to $B$ and $A$ is undecidable, then so is $B$.}
\end{equation}
}

The template (\ref{reduce-1}) allows us to set up proofs by contradiction to
prove undecidability.
\es

\bs{$A_{TM}$}
Recall the $A_{TM}$ language:

{\small
\fframe{{\bf Theorem:} The language
\[
A_{TM} = \{ \encode{M,w} | \mbox{$M$ is a TM and $M$ accepts $w$.}\}
\]
is undecidable.
}
}
\es

\bs{Reducibility}
{\small
\fframe{{\bf Theorem:} The language
\[
HALT_{TM} = \{ \encode{M,w} | \mbox{$M$ is a TM and $M$ halts on input $w$}\}
\]
is undecidable.
}
{\bf Proof:} Proof by contradiction.  We assume that $HALT_{TM}$ is decidable.
We show that $A_{TM}$ is reducible to $HALT_{TM}$ by constructing a machine
based on $HALT_{TM}$ that will decide $A_{TM}$.

Let $Q$ be a TM that decides $HALT_{TM}$. The we can construct a decider
$S$ that decides $A_{TM}$ as follows,
\begin{quote}
$S$ = "On input $\encode{M,w}$, where $M$ is a TM and $w$ a string:
\begin{enumerate}
\item[1.] Run $Q$ on $\encode{M,w}$.
\item[2.] If $Q$ rejects, \reject.
\item[3.] If $Q$ accepts, simulate $M$ on $w$ until it halts.
\item[4.] If $M$ has accepted, \accept; if $M$ has rejected, \reject."
\end{enumerate}
\end{quote}
We have shown that $A_{TM}$ is undecidable, therefore this is a contradiction
and our assumption that $HALT_{TM}$ is decidable must be incorrect. $\Box$

}
\es

\bs{Properties of $L(M)$}
\fframe{{\bf Theorem:}
The language
\[
E_{TM} = \{ \encode{M} | \mbox{$M$ is a TM and $L(M) = \emptyset$}\}
\]
is undecidable.}

{\bf Proof:} By contradiction.  Assume $E_{TM}$ is decidable and $Q$ is the decider.
We show that $A_{TM}$ reduces to $E_{TM}$ by constructing the following
decider $S$ for $A_{TM}$,
\begin{quote}
$S$ = "On input $\encode{M,w}$, where $M$ is a TM and $w$ a string:
\begin{enumerate}
\item[1.] Build the machine $M_1$ as follows,\\
$M_1$ = "On input $x$:
\begin{enumerate}
\item[1.] If $x \ne w$, \reject.
\item[2.] If $x = w$, run $M$ on input $w$ and \accept $\;$ if $M$ does."
\end{enumerate}
\item[2.] Run $Q$ in $\encode{M_1}$.
\item[3.] If $Q$ accepts, \reject; if $Q$ rejects, \accept."
\end{enumerate}
\end{quote}

But this machine cannot exist, therefore our assumption must be wrong.
\es



\bs{Properties of $L(M)$}
\fframe{{\bf Theorem:}
The language
\[
EQ_{TM} = \{ \encode{M_1,M_2} | \mbox{$M_1$ and $M_2$ are TMs and $L(M_1) = L(M_2)$}\}
\]
is undecidable.}

{\bf Proof:} By contradiction.  Assume $EQ_{TM}$ is decidable and $Q$ is the decider.
We show that $E_{TM}$ reduces to $EQ_{TM}$ by constructed the following
decider $S$ for $E_{TM}$,
\begin{quote}
$S$ = "On input $\encode{M}$, where $M$ is a TM:
\begin{enumerate}
\item[1.] Run $Q$ on input $\encode{M,M'}$ where  $M'$ is a TM that rejects
all inputs.
\item[2.] If $Q$ accepts, \accept; if $Q$ rejects, \reject."
\end{enumerate}
\end{quote}
But this machine cannot exist since $E_{TM}$ is undecidable, therefore our assumption must be wrong.

\es


\bs{Rice's Theorem}
{\small
In general,

\fframe{{\bf Theorem:} Testing any property of languages recognized by Turing machines is undecidable.}

{\bf Proof:} By contradiction. Let $P$ be a non-trivial property, then we want to show that
\[
L_P = \{ \encode{M} | \mbox{$L(M)$ satisfies $P$}\},
\]
is undecidable. \footnote{By non-trivial we mean that $L_P \ne \emptyset$ nor does it contain all TM's.}
Assume that $L_P$ is decidable and $M_P$ is a decider.
We now show that we can construct a decider $S$ for $A_{TM}$.

\begin{quote}
$S$ = "On input $\encode{M,w}$, where $M$ is a TM and $w$ a string:
\begin{enumerate}
\item Use $M$ and $w$ to construct the following TM $M'$:\\
$M'$ = "On input $x$:
\begin{enumerate}
\item Simulate $M$ on $w$.  If it halts and rejects, \reject. If it accepts, proceed
to stage (b).
\item \label{next} Simulate some $T$ on $x$, where $\encode{T}\in L_P$.  If it accepts, \accept." \footnote{Because $L_P$
is not trivial some $\encode{T} \in L_P$ has to exist.}
\end{enumerate}
\item Use $M_P$ to determine whether $\encode{M'} \in L_P$.
If YES, \accept. If NO, \reject."
\end{enumerate}
\end{quote}

It is easy to see that $\encode{M'}\in L_P$ iff $M$ accepts $w$, because $\encode{T}\in L_P$.
Since $A_{TM}$ is not decidable, this machine cannot exist and our
assumption that $L_P$ is decidable must be incorrect. $\Box$


}
\es


\bs{Mapping Reducibility}
Mapping Reducibility $\Rightarrow$ an computational approach to problem reduction.

\fframe{{\bf Definition:} A function $f\co \Sigma^* \rightarrow \Sigma^*$ is a 
{\bf\em computable function} if some Turing machine $M$, on every input $w$,
halts with just $f(w)$ on its tape.}

This allows us to formally define mapping reducibility,

\fframe{{\bf Definition:} Language $A$ is {\bf\em mapping reducible} to language
$B$, written $A \le_m B$, if there is a computable function $f\co \Sigma^* \rightarrow \Sigma^*$, where for every $w$,
\[
w \in A \Leftrightarrow f(w) \in B.
\]
The function $f$ is call the {\bf\em reduction} from $A$ to $B$.}

{\bf Observation:} The function $f$ does not have to be a correspondence (neither one-to-one nor 
surjective).  But, it is not allowed to map elements $w\not\in A$ into $B$.
\es

\bs{Mapping Reducibility}
\fframe{{\bf Theorem:} If $A \le_m B$ and $B$ is decidable, then $A$ is decidable.}

{\bf Proof:} Let $M$ be a decider for $B$ and let $f$ be a reduction from $A$ to $B$,
then we can construct a decider $N$ for $A$ as follows:

\begin{quote}
$N$ = "On input $w$:
\begin{enumerate}
\item[1.] Compute $f(w)$.
\item[2.] Run $M$ on $f(w)$ and output whatever $M$ outputs."
\end{enumerate}
\end{quote}

Clearly, $w \in A$ if $f(w) \in B$ since $f$ is a reduction.$\Box$
\es


\bs{Mapping Reducibility}
\fframe{{\bf Corollary:} If $A \le_m B$ and $A$ is undecidable, then $B$ is undecidable.}

{\bf Proof:} Assume that $B$ is decidable, let $M$ be a decider for $B$ and let $f$ be a reduction from $A$ to $B$,
then we can construct a decider $N$ for $A$ as follows:

\begin{quote}
$N$ = "On input $w$:
\begin{enumerate}
\item[1.] Compute $f(w)$.
\item[2.] Run $M$ on $f(w)$ and output whatever $M$ outputs."
\end{enumerate}
\end{quote}

But, since $A$ is undecidable by assumption this machine cannot exist and therefore
our assumption that $B$ is decidable must be wrong.$\Box$
\es


\bs{The Halting Problem II}
Let $HALT_{TM} = \{ \encode{M,w} | \mbox{$M$ is a TM and halts on $w$}\}.$  
We construct a reduction from $A_{TM}$ to $HALT_{TM}$ such that
\[
\encode{M,w} \in A_{TM} \Leftrightarrow \encode{M',w} \in HALT_{TM}.
\]
The following machine $F$ computes the reduction:
\begin{quote}
$F$ = "On input $\encode{M,w}$:
\begin{enumerate}
\item[1.]  Construct the following machine $M'$:\\
$M'$ = "on input $x$:
\begin{enumerate}
\item[1.] Run $M$ on $x$.
\item[2.] If $M$ accepts, \accept.
\item[3.] if $M$ rejects, loop."
\end{enumerate}
\item[2.] Output $\encode{M',w}$."
\end{enumerate}
\end{quote}
Observe that $\encode{M',w} \in HALT_{TM}$ if and only if $\encode{M,w} \in A_{TM}$ as required.
If the input to $F$ is not an element of $A$ we assume that $F$ maps it into some
string not in $B$.
\es

\bs{Reductions}
\vspace{.5in}

{\bf Observation:} Reductions between languages do not always exist.  That is, it is not
always possible to specify a computable function that reduces one language to another.
\es


\end{document}
%%%%%%%%%%%%%%%%%%%%%%%%%%% end of template1.tex %%%%%%%%%%%%%%%%%%%%%%%%%%%%%%%%

